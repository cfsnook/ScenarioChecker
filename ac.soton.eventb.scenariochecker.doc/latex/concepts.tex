% !TEX root = user_manual.tex

\section{Concepts}
\label{sec:concepts}

The Scenario Checker is based on the following concepts:
\begin{itemize}
		\item A model represents a closed system involving a controller and the sensed and controlled environment.
		\item A scenario consists of a sequence of external events that occur in the sensed environment. 
		\item After each external event, a resulting state is expected.
		\item The model of the controller may include internal events which are not specified in the scenario. 
		Events in the model may be designated as internal by adding a generic boolean attribute, \texttt{Internal=true}, or inserting \texttt{<INTERNAL>} in the comment property.
		\item The model of the controller may include private variables which are not specified in the scenario.
		Variables in the model may be designated as private by adding a generic boolean attribute, \texttt{Private=true}, or inserting \texttt{<PRIVATE>} in the comment property.
		\item Events may be given a priority to add control of ordering of events when this is not specified by the model. 
		Event priority is given by adding a generic integer attribute, \texttt{Priority=n}, or inserting \texttt{<PRIORITY=n>} in the comment property, where n is a natural number.
		(Low priority numbers are executed first).
		Note that prioritising events results in validating a particular refinement of the actual model.
\end{itemize}
	
The Scenario Checker provides:
\begin{itemize}
	\item A Control Panel to control the execution of external events, recording and playback of scenarios, and to view the enabled external events.
	\item A State View to monitor the value of variables and to indicate differences from the expected values.
	\item A Console View to show the history of execution and any other user feedback about the success or failure of the execution.
\end{itemize}


%%% Local Variables:
%%% mode: latex
%%% TeX-master: "user_manual"
%%% End:
